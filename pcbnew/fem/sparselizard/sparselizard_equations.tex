\documentclass[10pt]{article}         %% What type of document you're writing.

%%%%% Preamble

%% Packages to use

\usepackage{amsmath,amsfonts,amssymb}   %% AMS mathematics macros

%% Title Information.

\title{Basic \LaTeX{} Template}
\author{}
%% \date{2 July 2004}           %% By default, LaTeX uses the current date

%%%%% The Document

\begin{document}
%
\section{ Derivations from Maxwell Equations }
\[ \mathrm{B}  = \nabla \times \mathrm{A}\]
\[ \mathrm{E}  = - \nabla \cdot \mathrm{v}  - \frac{\partial  \mathrm{A}  }{ \partial t }\]
\subsection{Gauss' Law}
\[ \nabla \cdot \mathrm{E} = \frac{\rho}{\epsilon} \]
\[ \nabla \cdot \left( - \nabla \cdot \mathrm{v}  - \frac{\partial  \mathrm{A}  }{ \partial t } \right) = \frac{\rho}{\epsilon} \]
The divergence is a linear operator :
\[ - \nabla ^2  \cdot \mathrm{v} - \nabla \cdot \frac{\partial  \mathrm{A}  }{ \partial t }  = \frac{\rho}{\epsilon} \]

We now multiply both terms by a test function $\mathrm{X}$, and integrate it over a domain $\Omega$:

\[ -\int_{\Omega} \epsilon \cdot \left( \nabla ^2 \cdot \mathrm{v} \right) \cdot \mathrm{X} \cdot \mathrm{d}\Omega - \int_{\Omega} \epsilon \nabla \cdot \frac{\partial  \mathrm{A}  }{ \partial t } \cdot \mathrm{X} \cdot  \mathrm{d}\Omega  = \int_{\Omega} \rho \cdot \mathrm{X} \cdot \mathrm{d} \Omega  \]
\[ -\int_{\Omega} \epsilon \cdot \nabla  \mathrm{v}  \cdot \nabla \mathrm{X} \cdot \mathrm{d}\Omega  - \int_{\Omega} \epsilon  \nabla \cdot \frac{\partial  \mathrm{A}  }{ \partial t } \cdot \mathrm{X} \cdot  \mathrm{d}\Omega = \int_{\Omega} \rho \cdot \mathrm{X} \cdot \mathrm{d} \Omega  \]


\subsubsection{ In a conductor }
In a conductor, a common approximation is the current density being proportional to the electric field, $ J = \sigma \cdot E $. $\sigma$ being the conductivity of the conductor.

The law equivalent to Gauss' law in conductors becomes :

\[ \nabla \cdot \mathrm{J} = 0 \]
\[ \nabla \cdot \mathrm{ \sigma E } = 0 \]
\[  \sigma \nabla \cdot \left( - \nabla \cdot \mathrm{v}  - \frac{\partial  \mathrm{A}  }{ \partial t } \right)  = 0 \]

Thus,

\[ -\int_{\Omega} \sigma \cdot \nabla  \mathrm{v}  \cdot \nabla \mathrm{X} \cdot \mathrm{d}\Omega  - \int_{\Omega} \sigma  \nabla \cdot \frac{\partial  \mathrm{A}  }{ \partial t } \cdot \mathrm{X} \cdot  \mathrm{d}\Omega = 0  \]
\subsubsection{ In a conductor -DC }

\[ -\int_{\Omega} \sigma \cdot \nabla  \mathrm{v}  \cdot \nabla \mathrm{X} \cdot \mathrm{d}\Omega = 0  \]
\subsubsection{ In a dielectric  }
There is no net charge in a dielectric. ( That is, in a PCB )
\[ -\int_{\Omega} \epsilon \cdot \nabla  \mathrm{v}  \cdot \nabla \mathrm{X} \cdot \mathrm{d}\Omega  - \int_{\Omega} \epsilon  \nabla \cdot \frac{\partial  \mathrm{A}  }{ \partial t } \cdot \mathrm{X} \cdot  \mathrm{d}\Omega = 0 \]
\subsubsection{ In a dielectric - DC  }
\[ \int_{\Omega} \epsilon \cdot \nabla  \mathrm{v}  \cdot \nabla \mathrm{X} \cdot \mathrm{d}\Omega = 0  \]

\subsection{Gauss' Law for Magnetism}
\[ \nabla \cdot \mathrm{B} = 0 \]
\[ \nabla \cdot \mathrm{\nabla \times \mathrm{A}} = 0 \]
The divergence of a curl is 0, hence :
\[ 0 = 0 \]
Therefore, we cannot get anything useful from this law
\subsection{Lenz-Faraday's Law}
\[ \nabla \times \mathrm{E} = - \frac{ \partial \mathrm{B} }{ \partial t } \]
\[ \nabla \times \left( - \nabla \cdot \mathrm{v}  - \frac{\partial  \mathrm{A}  }{ \partial t } \right) = - \frac{ \partial \left( \nabla \times \mathrm{A} \right) }{ \partial t } \]
\[ \nabla \times  \nabla \cdot \mathrm{v}  - \frac{ \partial \left( \nabla \times \mathrm{A} \right) }{ \partial t } = - \frac{ \partial \left( \nabla \times \mathrm{A} \right) }{ \partial t } \]
\[ \nabla \times  \nabla \cdot \mathrm{v}  = 0 \]
The curl of a gradient is 0, hence :
\[ 0 = 0 \]
Therefore, we cannot get anything useful from this law


\subsection{Ampere's Law}
\[ \nabla \times \mathrm{B} = \mu \left(  \mathrm{ J }  + \epsilon \frac{ \partial \mathrm{E} }{ \partial t } \right) \]
\[ \nabla \times ( \nabla \times \mathrm{A} ) = \mu \left(  \mathrm{ J }  + \epsilon \frac{ \partial \left( - \nabla \cdot \mathrm{v}  - \frac{\partial  \mathrm{A}  }{ \partial t } \right) }{ \partial t } \right) \]
\[ \frac{1}{\mu} \nabla \times (\nabla \times \mathrm{A} ) =   \mathrm{ J }  - \epsilon \frac{ \partial \left(  \frac{\partial  \mathrm{A}  }{ \partial t } \right) }{ \partial t } - \epsilon \frac{ \partial \left(  \nabla \cdot \mathrm{v}  \right) }{ \partial t } \]

\[ \frac{1}{\mu}  \nabla \times ( \nabla \times \mathrm{A} ) =   \mathrm{ J }  - \epsilon \frac{ \partial ^2 A }{ \partial ^2 t } - \epsilon \nabla \cdot \frac{ \partial \left( \mathrm{v}  \right)
 }{ \partial t } \]

We now multiply both terms by a test function $\mathrm{X}$, and integrate it over a domain $\Omega$:

\[ \int_{\Omega} \frac{1}{\mu} \nabla \times ( \nabla \times \mathrm{A} ) \cdot X \cdot {d}\Omega =  \int_{\Omega} J \cdot X \cdot {d}\Omega - \int_{\Omega}   \epsilon \frac{ \partial ^2 A }{ \partial ^2 t } \cdot X \cdot {d}\Omega - \int_{\Omega} \epsilon \nabla \cdot \frac{ \partial \left( \mathrm{v}  \right)
}{ \partial t } \cdot X \cdot {d}\Omega  \]

\[ \int_{\Omega} \frac{1}{\mu} \nabla \times ( \nabla \times \mathrm{A} ) \cdot X \cdot {d}\Omega =  \int_{\Omega} J \cdot X \cdot {d}\Omega - \int_{\Omega}   \epsilon \frac{ \partial ^2 A }{ \partial ^2 t } \cdot X \cdot {d}\Omega - \int_{\Omega} \epsilon \nabla \cdot \frac{ \partial \left( \mathrm{v}  \right)
}{ \partial t } \cdot X \cdot {d}\Omega  \]

\subsubsection{In a conductor - General case}
In a conductor, a common approximation is the current density being proportional to the electric field, $ J = \sigma \cdot E $. $\sigma$ being the conductivity of the conductor.
\\
\[ \int_{\Omega} \frac{1}{\mu} \nabla \times ( \nabla \times \mathrm{A} ) \cdot X \cdot {d}\Omega =  - \int_{\Omega} \sigma \nabla v \cdot X \cdot {d}\Omega  - \int_{\Omega} \sigma  \frac{\partial  \mathrm{A}  }{ \partial t } \cdot X \cdot {d}\Omega - \int_{\Omega}   \epsilon \frac{ \partial ^2 A }{ \partial ^2 t } \cdot X \cdot {d}\Omega - \int_{\Omega} \epsilon \nabla \cdot \frac{ \partial \left( \mathrm{v}  \right)
}{ \partial t } \cdot X \cdot {d}\Omega  \]

We could even add that in a good conductor, there is little to no displacement current:

\[ \int_{\Omega} \frac{1}{\mu} \nabla \times ( \nabla \times \mathrm{A} ) \cdot X \cdot {d}\Omega =  - \int_{\Omega} \sigma \nabla v \cdot X \cdot {d}\Omega  - \int_{\Omega} \sigma  \frac{\partial  \mathrm{A}  }{ \partial t } \cdot X \cdot {d}\Omega \]
\subsubsection{In a conductor - DC}
\[ \int_{\Omega} \frac{1}{\mu} \nabla \times ( \nabla \times \mathrm{A} ) \cdot X \cdot {d}\Omega =  - \int_{\Omega} \sigma \nabla v \cdot X \cdot {d}\Omega    \]
\subsubsection{In a conductor - DC - Ignoring magnetic field}
\[ \int_{\Omega} \sigma \nabla v \cdot X \cdot {d}\Omega = 0    \]

\subsubsection{In a dielectric}
There cannot be any current flowing in a good dielectric, hence $\mathrm{ J } = 0 $
\[ \int_{\Omega} \frac{1}{\mu} \nabla \times ( \nabla \times \mathrm{A} ) \cdot X \cdot {d}\Omega = \epsilon \frac{ \partial ^2 A }{ \partial ^2 t } \cdot X \cdot {d}\Omega - \int_{\Omega} \epsilon \nabla \cdot \frac{ \partial \left( \mathrm{v}  \right)
}{ \partial t } \cdot X \cdot {d}\Omega  \]


\subsubsection{In a dielectric - DC}
\[ \int_{\Omega} \frac{1}{\mu} \nabla \times ( \nabla \times \mathrm{A} ) \cdot X \cdot {d}\Omega =0 \]

\end{document}

